\documentclass{beamer}
\usepackage[utf8]{inputenc}
\usepackage{color}
\usepackage{subcaption}
\usepackage{tikz}
\usetikzlibrary{graphs,positioning}
\usepackage{lmodern}

\usepackage{amsmath,amssymb,amsthm}    
\usepackage{mathabx}\changenotsign   
\usepackage{mathrsfs} 
\usepackage{dsfont} 
\usepackage[babel]{microtype}
\usepackage{xcolor}  	
\usepackage[export]{adjustbox}

\usepackage[brazil]{babel}   
\usepackage[utf8]{inputenc} 
\usepackage{comment}
\usepackage{bbm}

\newtheorem{thm}[equation]{Teorema}
\newtheorem{cor}[equation]{Corolário}
\newtheorem{lem}[equation]{Lema}
\newtheorem{prop}[equation]{Proposição}
\newtheorem{conj}[equation]{Conjectura}


\def\moverlay{\mathpalette\mov@rlay}
\def\mov@rlay#1#2{\leavevmode\vtop{   \baselineskip\z@skip \lineskiplimit-\maxdimen
		\ialign{\hfil$\m@th#1##$\hfil\cr#2\crcr}}}
\newcommand{\charfusion}[3][\mathord]{
	#1{\ifx#1\mathop\vphantom{#2}\fi
		\mathpalette\mov@rlay{#2\cr#3}
	}
	\ifx#1\mathop\expandafter\displaylimits\fi}
\makeatother

\DeclareMathOperator{\dom}{{\rm dom}}

\newcommand{\dcup}{\charfusion[\mathbin]{\cup}{\cdot}}
\newcommand{\bigdcup}{\charfusion[\mathop]{\bigcup}{\cdot}}


\def\ra{\longrightarrow}
\def\dom{\text{\rm dom}}

\def\R{\text{\textcolor{red}{\rm red}}}
\def\B{\text{\textcolor{blue}{\rm blue}}}


\def\mcarrow{\xrightarrow[{\raisebox{.5mm}[1mm][0mm]{$\scriptstyle \rm p$}}]{\raisebox{0.0mm}[0mm]{$\scriptstyle \rm mc$}}}
\def\pmc#1{p^{\rm mc}_{#1}}

\def\red{\text{\textcolor{red}{\rm red}}}
\def\blue{\text{\textcolor{blue}{\rm blue}}}
\def\green{\text{\textcolor{gree}{\rm green}}}

\tikzset{onslide/.code args={#1#2}{%
    \only<#1>{\pgfkeysalso{#2}}
}}

\newtheorem{teorema}             {Teorema}       
\newtheorem{Afirmativa}[teorema] {Claim}         
\newtheorem{lema}      [teorema] {Lema}         
\newtheorem{corolario} [teorema] {Corolário}     
\newtheorem{fato}      [teorema] {Fato}    
\newtheorem{proposicao}      [teorema] {Proposição}                
\newtheorem{conjectura}[teorema] {Conjectura}    
\newtheorem{problema}  [teorema] {Problema}       
\newtheorem{definicao}  [teorema] {Definição}       









\title[Combinatória Extremal]{PGC \\ Combinatória Extremal}

\author[Diogo Alves - UFABC]{\textbf{Diogo Eduardo Lima Alves}\\ \ \\ \ \\ \ \\ \ \\Universidade Federal do ABC - UFABC\\ \ \\ \ \\ 
}

\date{9 de maio de 2019}

\usetheme{Madrid}
\usecolortheme{beaver}


\setbeamertemplate{part page}{
        \begin{beamercolorbox}[sep=15pt,center,wd=\textwidth]{part title}
            \usebeamerfont{part title}\insertpart\par
        \end{beamercolorbox}
}


\begin{document}
\maketitle

\part{Combinatória Extremal}

\frame{\partpage}

\frame{
	\frametitle{Combinatória Extremal}

	\begin{itemize}
	\item		\textbf{Combinatória Extremal}: Subtema da combinatória  que estuda quão grande ou pequena uma estrutura pode ser ao mesmo tempo que satisfaz certas condições.\vspace{0.2cm}
	\end{itemize}
}

\frame{
	\frametitle{Combinatória Extremal}

	Exemplos:\vspace{0.2cm}
	\begin{itemize}
		\item Qual a maior quantidade de arestas que um grafo $G$ pode ter, sem que $G$ tenha um subgrafo $H$?\vspace{0.2cm}\pause
		
		\item Qual o tamanho do maior conjunto independente em um grafo? (NP-completo)
					
	\end{itemize}
}

\frame{
	\frametitle{\'Areas Abordadas}

	\begin{itemize}
	\item Teoria de Ramsey
	\item Grafos Extremais
	\item Grafos Aleat\'orios
	\item Regularidade
	\end{itemize}
}

\frame{
	\frametitle{Teoria de Ramsey}
	\begin{theorem}[{Schur's Theorem}]\label{thm:Schur'sTheorem} %Schur, 1916
Para toda colorac\~ao $c\colon \mathbb{N} \rightarrow [r]$, existe $x,y,z$ tal que $x+y=z$ e $c(x) = c(y) = c(z)$.
\end{theorem}
}
\frame{
	\frametitle{Teoria de Ramsey}
\begin{proof}
  Para a colorac\~ao de vertices dada por $c\colon \mathbb{N} \rightarrow [r]$ defina uma colorac\~ao de arestas dada por $c'\colon \binom{\mathbb{N}}{2} \rightarrow [r]$ como $c'(\{a,b\}) \colon= c(|a-b|)$. Pelo Teorema de Ramsey sabemos que existe um tri\^angulo monocrom\'atico, assuma que $\{x,y,z\}$ forma o tri\^angulo, com $x<y<z$.\\
Usando a definic\~ao de $c'$  temos:
$$c'(\{x,y\}) = i = c(|y-x|)$$
$$c'(\{x,z\}) =  i = c(|z-x|)$$
$$c'(\{y,z\}) = i = c(|z-y|).$$

Segue $c(|y-x|) = c(|z-x|) = c(|z-y|)$, e $(z-y)+(y-x)=(z-x)$ que implica $x,y,z$ tal que $x+y=z$ e $c(x) = c(y) = c(z)$, como requisitado.\\
\end{proof}
}

\frame{
	\frametitle{Grafos Extremais}
		\begin{theorem}[{Erd\H{o}s}] \label{theorem: Erdos,1938}
Para todo grafo $G$ com $n$ vertices
$$ex(n,C_4) = O(n^{3/2}).$$
\end{theorem}
O $C_4$ \'e formado por duas `cerejas' no mesmo par de vertices. Contando essas triplas (x,\{y,z\}) em $G$ tal que $xy, xz \in E(G)$ e usando a inequac\~ao de Jensen com $\lambda_i = 1/n$ obtemos:
	$$ \sum_{i=1}^n \frac{1}{n} f\left(x_i\right) \geq f		\left(\sum_{i=1}^n \frac{1}{n} x_i\right).$$
}
\frame{
	\frametitle{Grafos Extremais}
Aplicando ao nosso problema temos,
$$ \frac{\sum_{i=1}^n \binom{x_i}{2}}{n} \geq \binom{\frac{\sum_{i=1}^n x_i}{n}}{2} ,$$
Substituindo $x_i$ e lembrando que $\sum_{v \in V(G)} d(v) = 2e(G),$
\begin{align*}
\sum_{v \in V(G)} \binom{d(v)}{2} &\geq n \binom{\frac{2e(G)}{n}}{2}\\
&= n\frac{\frac{2e(G)}{n}\left( \frac{2e(G)}{n}-1\right)}{2} \\
&\geq \frac{n}{2} \left( \frac{2e(G)}{n} - 1 \right)^2.
\end{align*}
}
\frame{
	\frametitle{Grafos Extremais}
Note que o n\'umero m\'aximo dessas triplas em um grafo $C_4$-livre \'e no m\'aximo $\binom{n}{2}$ porque podemos ter apenas uma cereja em cada par de vertices.
$$ \frac{n}{2}\left(\frac{2e(G)}{n} - 1\right)^2 \leq \binom{n}{2},$$
Temos $e(G) = O(n^{3/2})$ terminando a prova.
}

\frame{
	\frametitle{Grafos Aleat\'orios}

		\begin{theorem}[Chebyshev's Inequality]
		$$\mathbb{P}(|X-\mu| \geq a) \leq \sigma^2/a^2.$$
		\end{theorem}
		Usando $a = \mu$, temos a seguinte inequac\~ao,

		$$\mathbb{P}(X=0) \leq \mathbb{P}(|X-\mu | \geq \mu) 		\leq \sigma^2/\mu^2,$$
		que nos d\'a um limitante superior para $\mathbb{P}(X=0)$.\\
	
}
\frame{
	\frametitle{Grafos Aleat\'orios}
	\begin{theorem}
		Seja $G=G(n,p)$ um grafo aleat\'orio.
		Ent\~ao,
		$$
		\mathbb{P}(G\text{ conter um tri\^angulo}) \rightarrow 
		\begin{cases}
			0, &\text{if $p\ll 1/n$},\\
			1, &\text{if $p\gg 1/n$}\,.
		\end{cases}
		$$
	\end{theorem}
	\begin{itemize}
	\item Prova:
	$$X = \text{ quantidade de tri\^angulos em }G$$
	\begin{align*}
	\mathbb{P}(G\text{ conter um tri\^angulo}) &\leq \mathbb{E}(X)\\
	& \leq \binom{n}{3}p^3 \\
	& \ll 1,
\end{align*}
	se $p \ll 1/n$.\begin{align*}
\end{align*} 
	\end{itemize}			
}

\frame{
	\frametitle{Grafos Aleat\'orios}
\begin{align*}
Var(X) &= \mathbb{E}(X^2) - \mathbb{E}(X)^2\\
& = \mathbb{E} \left( \sum_{(u, v)} \mathbbm{1}[u]\mathbbm{1}[v] \right) - \left( \sum_u \mathbb{P}(u) \right) ^2\\
& = \sum_{u,v} \left(\mathbb{P}(u \wedge v) - \mathbb{P}(u)\mathbb{P}(v)\right),
\end{align*}
usando a inequac\~ao de Chebychev,
$$\mathbb{P}(G \text{ n\~ao conter tri\^angulos}) \leq \frac{Var(X)}{\mathbb{E}(X)^2} \leq \frac{n^4p^5 + n^3p^3 }{n^6p^6} \ll 1,$$
para $p \gg 1/n$, terminando a prova.
}

\frame{
	\frametitle{Regularidade}
	\begin{definition}
Given a graph $G$ and disjoint sets $A$ and $B$ of vertices, we say that $(A,B)$ is $\varepsilon$-regular if $\text{ for every } X \subset A \text{ and every } Y \subset B\text{ with } |X| \geq \varepsilon|A| \text{ and } |Y| \geq \varepsilon |B|$ we have
$$ \left| \frac{e(X,Y)}{|X||Y|} - \frac{e(A,B)}{|A||B|} \right| \leq \varepsilon  .$$
\end{definition}
	
	\begin{lemma}\label{lemma:embeddinglemma}
(The Embedding Lemma - simple version). Let $H$ be a graph, and let $\varepsilon >0$. There exist $\delta >\varepsilon$ and $M \in \mathbb{N}$ such that if $m \geq M$ and there exist a partition $\{V_1, ..., V_H\}$ with all pairs being $\varepsilon$-regular and $\delta$-dense, then $H \subset G$. 
\end{lemma} 
}

\frame{
	\frametitle{Regularidade}
\begin{theorem}[{The Szemerédi Regularity Lemma \cite{Sz75}}]. Let $\varepsilon > 0$, and let $m \in  \mathbb{N}$. There exists a constant $M=M(m,\varepsilon)$ such that the following holds.\\
For any graph $G$, there exists a partition $V(G) = \{V_0 \cup ... \cup V_k\}$ of the vertex set into $m \leq k \leq M$ parts, such that
\begin{itemize}

	\item $|V_1| = ... =|V_k|$,
	
	\item $|V_0| \leq \varepsilon|V(G)|$,
%$|A_0| \leq \varepsilon |V(G)|$,
	\item all but $\varepsilon k^2$ of the pairs $(V_i, V_j)$ are $\varepsilon$-regular. 
\end{itemize}
\end{theorem}
}
\frame{
\begin{theorem}[{Triangle Removal Lemma}]
For all $\alpha > 0$ exists $\beta > 0$ such that if $G$ is a graph with $\leq \beta n^3$ triangles, then it is possible to remove all triangles removing at most $\alpha n^2$ edges. 
\end{theorem}

	\frametitle{Regularidade}
The first, second and third steps of the method are, in general, the same for classical problems,

1. Apply SzRL with $\varepsilon$ enough small and we have the partitions $\{V_1, ... , V_k\}$.

2. Remove edges inside the partitions, between irregular pairs and sparse pairs obtaining $G'$ with at most $\alpha n^2$ edges removed.

3. Define $R$ with $V(R) = [k]$ and $\{i,j\} \in E(R) $ if  the pair $(A_i, A_j)$ is dense and $\varepsilon$-regular.
}

\frame{
	\frametitle{Regularidade}
	
Now we have two cases, in the first one we have a triangle in $R$. Note if $\beta n^3 < 1$ the result is trivial, then we assume $\beta \geq 1/n^3$ and the partitions that forms the triangle (assume $\{V_1, V_2, V_3\}$ for simplicity) has size $n/k \geq 1/(\beta ^{1/3} k) \geq m$.

Applying the Embedding Lemma we have that the quantity of triangles in $G'$ is at least $\delta ^3 /2 (n/k)^3 > \beta n^3$ if we choose $\beta$ such that $\delta ^3 / (2k^3) > \beta$.

We conclude that if $\alpha n^2$ edges are removed and the graph still has triangles the triangles quantity is more than $\beta n^3$.

In the second case there is no triangle in $R$ and this implies there is no triangle in $G'$ because the edges inside the pairs $\{V_1, ... , V_k\}$ were removed then the only possible triangles are formed between the pairs, finishing the proof.
}


\frame{
	\frametitle{Agradecimentos}
	\begin{center}
	Obrigado!
	\end{center}
}	
\end{document}
